\chapter{Числовые представления}
\label{ch:9}

Рассмотрим обыкновенные представления списков и натуральных чисел, а
также несколько типичных функций над этими типами данных.
\begin{lstlisting}
  datatype $\alpha$ List =                   datatype Nat =
       Nil                                Zero
     | Cons of $\alpha$ $\times$ $\alpha$ List                  | Succ of Nat

  fun tail (Cons (x, xs)) = xs       fun pred (Succ n) = n

  fun append (Nil, ys) = ys          fun plus (Zero, n) = n
    | append (Cons (x, xs), ys) =      | plus (Succ m, n) =
       Cons (x, append (xs, ys))          Succ (plus (m, n))
\end{lstlisting}
Помимо того, что списки содержат элементы, а натуральные числа нет,
эти две реализации практически совпадают. Подобным же образом
соотносятся биномиальные кучи и двоичные числа. Эти примеры наводят на
сильную аналогию между представлениями числа $n$ и представлениями
объектов-контейнеров размером $n$. Функции, работающие с контейнерами,
сильно напоминают арифметические функции, работающие с
числами. Например, добавление нового элемента напоминает увеличение
числа на единицу, удаление элемента напоминает уменьшение числа на
единицу, а слияние двух контейнеров напоминает сложение двух
чисел. Можно использовать эту аналогию для проектирования новых
представлений абстракций контейнеров~--- достаточно выбрать
представление натуральных чисел, обладающее заданными свойствами и
соответствующим образом определить функции над
объектами-контейнерами. Назовем реализацию, спроектированную таким
образом, \term{числовым представлением}{numerical representation}.

В этой главе мы исследуем несколько числовых представлений для двух
различных абстракций: \term{кучи}{heaps} и \term{списков со свободным
  доступом}{random-access lists} (известных также как \term{гибкие массивы}{flexible
arrays}). Эти две абстракции подчеркивают различные наборы
арифметических операций. Для куч требуются эффективные функции
увеличения на единицу и сложения, а для списков со свободным доступом
требуются эффективные функции увеличения и уменьшения на единицу.

\section{Позиционные системы счисления}
\label{sc:9.1}

\term{Позиционная система счисления}{positional number system}
\cite{Knuth1973b}~--- способ записи числа в виде поаледовательности
цифр $b_0\ldots b_{m-1}$. Цифра $b_0$ называется \term{младшей}{least
  significant digit}, а цифра $b_{m-1}$ \term{старшей}{most
  significant digit}. Кроме обычных десятичных чисел, мы всегда будем
записывать последовательности цифр в порядке от младшей к старшей.

%%% Local Variables: 
%%% mode: latex
%%% TeX-master: "pfds"
%%% End: 
